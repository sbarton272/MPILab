
\title{15-440: Lab 4}
\author{
        Spencer Barton (sebarton)\\
        Emma Binns (ebinns)
}
\date{December 5, 2014}

\documentclass[12pt]{article}

\usepackage{graphicx}
\usepackage[compact]{titlesec}
\usepackage[letterpaper, portrait, margin=1in]{geometry}
\usepackage{amsmath, amsfonts, amsthm, amssymb}

\newcommand{\ttt}{\texttt}

\setlength{\parindent}{0pt}
\setlength{\parskip}{\baselineskip}

\begin{document}
\maketitle

%------------------------------------------
\pagebreak
\section{Running the Framework}

\subsection{System Requirements}

MPI runs on the Gates cluster machines. It requires OpenMPI be installed.

\subsection{Set-up}

The following steps must be followed for a successful set-up.

\begin{enumerate}
\item
TODO setup MPI

\item 
Run \ttt{python SETUP.py} in the \ttt{src} directory. This will compile all of the code.

\item
Set-up the configuration file to run your job. See below for how to create your own configuration file or use one of the examples in the \ttt{src/test} directory.
\end{enumerate}

\subsection{Run a Job}

Simply run TODO-command

\subsection{Configuration Files}

Most of the configuration file parameters are self-explanatory:

\begin{itemize}
\item
\ttt{JOB\_NAME=<string>}

\item
\ttt{INPUT\_FILE=<filePath>}

\item
\ttt{OUTPUT\_FILE=<filePath>}

\item
\ttt{KMEANS\_TYPE=<Sequential|Parallel>}

\item
\ttt{N\_CLUSTERS=<int>}

\item
\ttt{DATA\_TYPE=<DNA|XYPt>}

\item
\ttt{TEMINATION=<float>}

\end{itemize}

\subsection{Data Generation}

\subsection{Running Provided Examples}

To run the provided examples follow the above steps and use the provided config files. Make sure to modify the participants and number of participants as necessary. Take a look at the provided code for examples of how to write the map and reduce functions.

%------------------------------------------
\section{Project Requirements}

\subsection{Requirements Met and Capabilities}

\subsection{Requirements Not Met and Limitations}

\subsection{Improvements}

\end{document}
