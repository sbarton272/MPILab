
\title{15-440: Lab 4}
\author{
        Spencer Barton (sebarton)\\
        Emma Binns (ebinns)
}
\date{December 5, 2014}

\documentclass[12pt]{article}

\usepackage{graphicx}
\usepackage[compact]{titlesec}
\usepackage[letterpaper, portrait, margin=1in]{geometry}
\usepackage{amsmath, amsfonts, amsthm, amssymb}

\newcommand{\ttt}{\texttt}

\setlength{\parindent}{0pt}
\setlength{\parskip}{\baselineskip}

\begin{document}
\maketitle

%------------------------------------------
\pagebreak
\section{Running the Framework}

\subsection{System Requirements}

MPI runs on the Gates cluster machines. It requires OpenMPI be installed.

\subsection{Set-up}

The following steps must be followed for a successful set-up.

\begin{enumerate}
\item
TODO setup MPI

\item 
Run \ttt{python SETUP.py} in the \ttt{src} directory. This will compile all of the code.

\item
Set-up the configuration file to run your job. See below for how to create your own configuration file or use one of the examples in the \ttt{src/test} directory.
\end{enumerate}

\subsection{Run a Job}

Simply run TODO-command

\subsection{Configuration Files}

Most of the configuration file parameters are self-explanatory:

\begin{itemize}
\item
\ttt{JOB\_NAME=<string>}

\item
\ttt{INPUT\_FILE=<filePath>}

\item
\ttt{OUTPUT\_FILE=<filePath>}

\item
\ttt{KMEANS\_TYPE=<Sequential|Parallel>} We run k-means either in parallel or sequential mode. If parallel mode is specified then the algorithm runs across the specified participants, otherwise it runs just on the master.

\item
\ttt{N\_CLUSTERS=<int>} This specifies the number of clusters to search for. This is a design parameter and should be selected based off of your intuition.

\item
\ttt{DATA\_TYPE=<DNA|XY>} Our k-means currently supports two data types, \ttt{DNA} and \ttt{XY} points. Please specify which you will be using.

\item
\ttt{TEMINATION=<float>} The algorithm iterates until the means converge within the this termination threshold. This should be a positive number.

\end{itemize}

\subsection{Data Generation}
In the \ttt{DataGeneratorScripts} directory there is are two python scripts to generate data for testing. Each has a simply command line interface that can be determined by running the respective scripts. The script \ttt{generaterawdata.py} generates \ttt{XY} points and \ttt{generateDnaData.py} generates \ttt{DNA} data.

\subsection{Running Provided Examples}

To run the provided examples follow the above steps and use the provided configuration files. Take a look in each config file to see what parameters are utilized. These configuration files make use of the data in the \ttt{test} directory.

%------------------------------------------

\section{K-Means Implementation}

\subsection{Sequential}
The sequential algorithm works as specified in the handout:

\begin{enumerate}

\item
Init random means

\item
For all data organize by nearest mean

\item
For all means generate a new mean with the closest data points per mean

\item
Iterate on the new mean generation process until the means converge

\end{enumerate}

\subsection{Parallel}

The parallel algorithm is effectively the same except that the organizing of data points by nearest mean is a distributed process. Workers calculate new intermediate means based upon the data that they have and then send these means back to the master. The master combines all of these intermediate means and checks for convergence.

\begin{enumerate}

\item
[Master] Init random means

\item
[Master] Partition data out to the participants

\item
[Master] Send current means to participants

\item
[Participant] For all data organize by nearest mean

\item
[Participant] For all means generate a new mean with the closest data points per mean

\item
[Participant] Send these intermediate means back to the master

\item
[Master] Combine intermediate means and check for convergence. If not converged, send these new means out the the participants and repeat the above steps.

\end{enumerate}

Intermediate means are specific to the data involved. For \ttt{XYPt} objects the mean representation is a sum of the x components and a sum of the y components along with a element count such that the mean itself is \ttt{<sumX/numPts, sumY/numPts}. \ttt{sumX}, \ttt{sumY} and \ttt{numPts} can all be updated as new points are added. For \ttt{DNA} the mean representation is a count of the bases for each base in the dna strand. For example if the strand is \ttt{AGGT} the mean representation could be \ttt{<[3,0,0,1],[1,2,1,0],[1,2,0,1],[1,1,2,0]>} where each array represents the counts of \ttt{[A,G,T,C]}. As new dna strands are added to the mean these counts can be iterated appropriately.

%------------------------------------------

\section{Experimentation and Analysis}

TODO - multiple num processes over different data set sizes

\end{document}
